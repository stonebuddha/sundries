%% Copyright 2006-2013 Xavier Danaux (xdanaux@gmail.com).
%
% This work may be distributed and/or modified under the
% conditions of the LaTeX Project Public License version 1.3c,
% available at http://www.latex-project.org/lppl/.


\documentclass[11pt,letterpaper,roman]{moderncv}        % possible options include font size ('10pt', '11pt' and '12pt'), paper size ('a4paper', 'letterpaper', 'a5paper', 'legalpaper', 'executivepaper' and 'landscape') and font family ('sans' and 'roman')

% modern themes
\moderncvstyle{banking}                            % style options are 'casual' (default), 'classic', 'oldstyle' and 'banking'
\moderncvcolor{black}                                % color options 'blue' (default), 'orange', 'green', 'red', 'purple', 'grey' and 'black'
%\renewcommand{\familydefault}{\sfdefault}         % to set the default font; use '\sfdefault' for the default sans serif font, '\rmdefault' for the default roman one, or any tex font name
\nopagenumbers{}                                  % uncomment to suppress automatic page numbering for CVs longer than one page

% character encoding
\usepackage[utf8]{inputenc}                       % if you are not using xelatex ou lualatex, replace by the encoding you are using
%\usepackage{CJKutf8}                              % if you need to use CJK to typeset your resume in Chinese, Japanese or Korean
\usepackage{fourier}
\usepackage[osf]{Alegreya}
\usepackage{AlegreyaSans}
\usepackage[zerostyle=c]{newtxtt}
\linespread{1.04}

% adjust the page margins
\usepackage[scale=0.86]{geometry}
%\setlength{\hintscolumnwidth}{3cm}                % if you want to change the width of the column with the dates
%\setlength{\makecvtitlenamewidth}{10cm}           % for the 'classic' style, if you want to force the width allocated to your name and avoid line breaks. be careful though, the length is normally calculated to avoid any overlap with your personal info; use this at your own typographical risks...

\usepackage{import}

% personal data
\name{Di}{Wang}
% \title{Assistant Professor}                               % optional, remove / comment the line if not wanted
\address{Yanyuan Mansion 520, 151 Zhongguancun N Ave, Haidian District, Beijing 100084}{}{}% optional, remove / comment the line if not wanted; the "postcode city" and and "country" arguments can be omitted or provided empty
%\phone[mobile]{+86 138 8226 6228}                   % optional, remove / comment the line if not wanted
%\phone[fixed]{01234 123456}                    % optional, remove / comment the line if not wanted
%\phone[fax]{+3~(456)~789~012}                      % optional, remove / comment the line if not wanted
\email{wangdi95@pku.edu.cn}                               % optional, remove / comment the line if not wanted
\urlstyle{rm}
\renewcommand{\httpslink}[2][]{%
  \ifthenelse{\equal{#1}{}}%
  {\url{#2}}%
  {\href{https://#2}{\detokenize{#1}}}}
\homepage{https://stonebuddha.github.io/}                         % optional, remove / comment the line if not wanted
%\extrainfo{Carnegie Mellon University -- Computer Science Department -- Doctoral Student}                 % optional, remove / comment the line if not wanted
%\photo[64pt][0.4pt]{picture}                       % optional, remove / comment the line if not wanted; '64pt' is the height the picture must be resized to, 0.4pt is the thickness of the frame around it (put it to 0pt for no frame) and 'picture' is the name of the picture file
%\quote{Some quote}                                 % optional, remove / comment the line if not wanted

% to show numerical labels in the bibliography (default is to show no labels); only useful if you make citations in your resume
\makeatletter
\renewcommand*{\bibliographyitemlabel}{\@biblabel{\arabic{enumiv}}}
\makeatother
%\renewcommand*{\bibliographyitemlabel}{[\arabic{enumiv}]}% CONSIDER REPLACING THE ABOVE BY THIS

% bibliography with multiple entries
\usepackage{multibib}
\newcites{conf,thesis}{{Refereed Conference Papers},{Other Publications}}
%----------------------------------------------------------------------------------
%            content
%----------------------------------------------------------------------------------
\begin{document}
%\begin{CJK*}{UTF8}{gbsn}                          % to typeset your resume in Chinese using CJK
%-----       resume       ---------------------------------------------------------
\makecvtitle

%\small{Undergraduate electrical and electronic engineer completing the final year of a master's degree. Passionate about science, with strong technical, business, and interpersonal skills for working in a team and successfully completing a project.}

\section{Bio}

I am an Assistant Professor at Peking University.
%
I received my Ph.D. from Carnegie Mellon University under the supervision of
Prof. Jan Hoffmann.
%
My research focuses are programming languages,
quantitative verification,
and probabilistic programming;
my broader interests include type theory, program synthesis, concurrency, and
Bayesian inference.
%
During my Ph.D.,
I built an effective toolkit for rigorous and automatic analysis of probabilistic
programs (PLDI'18, MFPS'19, ICFP'20, PLDI'21),
the first coroutine-based paradigm for sound programmable Bayesian inference (PLDI'21),
the first sound and relatively complete worst-case input generation algorithm (POPL'19),
and the first resource-aware synthesizer for recursive programs (PLDI'19, ICFP'20).

\section{Education}

%\begin{itemize}
  {\cventry{Aug 2017 -- May 2022}{Ph.D. in Computer Science}{Carnegie Mellon University}{Pittsburgh, PA, USA}{}{Advisor: Prof.\ Jan Hoffmann \\ Thesis: \emph{Static Analysis of Probabilistic Programs: An Algebraic Approach}}}

  \vspace{4pt}

  {\cventry{Sep 2013 -- Jun 2017}{Bachelor of Science (with Honors) in Computer Science \& Technology}{Peking University}{Beijing, China}{}{GPA: 3.83/4.0 (\textbf{ranked 3rd} out of \textasciitilde 200) \\ Advisor: Prof.\ Yingfei Xiong \\ Thesis: \emph{Accelerating Program Analyses by Conditional Summarization with Datalog}}}
%\end{itemize}

\section{Research Experiences}

%\begin{itemize}
%  \item{\cventry{Aug 2017 -- present}{Research assistant, supervised by Prof. Jan Hoffmann}{Carnegie Mellon University}{Pittsburgh, PA}{}
%  {Topics: Static Analysis of Probabilistic Programs, Automatic Resource Bound Analysis}}

%  \vspace{4pt}

  {\cventry{May 2020 -- Aug 2020}{Research intern, supervised by Dr.\ Herman Venter}{Facebook}{Seattle, WA, USA}{}
  {Topics: Formal Verification of Rust Code, Side Channel Analysis of Blockchain Code}}

  \vspace{4pt}

  {\cventry{Sep 2016 -- Jan 2017}{Research intern, supervised by Prof.\ Adam Chlipala}{Massachusetts Institute of Technology}{Boston, MA, USA}{}
  {Topics: Type System for Complexity Analysis, Complexity Preserved Compiler}}

  \vspace{4pt}

  {\cventry{Jun 2016 -- Aug 2016}{Research intern, supervised by Prof.\ Thomas Reps}{University of Wisconsin--Madison}{Madison, WI, USA}{}
  {Topics:Probabilistic Reasoning about Side Channel Attacks, Expectation Invariant Analysis of Probabilistic Programs}}

  \vspace{4pt}

  {\cventry{Sep 2015 -- Jun 2017}{Research assistant, supervised by Prof.\ Lu Zhang and Prof.\ Yingfei Xiong}{Peking University}{Beijing, China}{}
  {Topics: Complete Library Summarization for Program Analyses, Pointer Analysis for Java}}
%\end{itemize}

% Publications from a BibTeX file without multibib
%  for numerical labels: \renewcommand{\bibliographyitemlabel}{\@biblabel{\arabic{enumiv}}}% CONSIDER MERGING WITH PREAMBLE PART
%  to redefine the heading string ("Publications"): \renewcommand{\refname}{Articles}
%\nocite{*}
%\bibliographystyle{plain}
%\bibliography{publications}                        % 'publications' is the name of a BibTeX file

% Publications from a BibTeX file using the multibib package
\section{Publications}
\nociteconf{PLDI:WHR21B,PLDI:WHR21A,ICFP:WKH20,ICFP:KWR20,PLDI:KWPH19,MFPS:WHR19,POPL:WH19,PLDI:WHR18,OOPSLA:WWC17,ESOP:TWX17}
\bibliographystyleconf{unsrt}
\bibliographyconf{publications}                   % 'publications' is the name of a BibTeX file
\vspace{4pt}
\nocitethesis{misc:DWH21,misc:WHR21}
\bibliographystylethesis{unsrt}
\bibliographythesis{publications}
%\vspace{4pt}
%\nocitedraft{misc:WWC18}
%\bibliographystyledraft{unsrt}
%\bibliographydraft{publications}
%\nocitemisc{misc1,misc2,misc3}
%\bibliographystylemisc{plain}
%\bibliographymisc{publications}                   % 'publications' is the name of a BibTeX file

\section{Teaching and Mentoring Experience}

\begin{itemize}
  \item \textbf{Guest Lecturer} -- \emph{Foundations of Quantitative Program Analysis}, Carnegie Mellon University \hfill 2019

  \vspace{4pt}

  \item \textbf{Teaching Assistant} -- \emph{Bug Catching: Automated Program Verification}, Carnegie Mellon University \hfill 2020
  \item \textbf{Teaching Assistant} -- \emph{Programming Language Semantics}, Carnegie Mellon University \hfill 2019
  \item \textbf{Teaching Assistant} -- \emph{Introduction to Computer Systems}, Peking University \hfill 2015

  \vspace{4pt}

  \item \textbf{Mentor} -- Vanshika Chowdhary, \emph{Programmable Gibbs sampling with linear types} \hfill 2021
  \item \textbf{Mentor} -- Mohamed Lotfi, \emph{Synthesis of probabilistic programs that generate handwritten digits} \hfill 2021
  \item \textbf{Mentor} -- Charles Yuan, \emph{Exact Bayesian inference with distribution transformers} \hfill 2019
\end{itemize}

\section{Professional Activities}

\begin{itemize}
  \item \textbf{Artifact Evaluation Committee Member} -- POPL'19, POPL'20, CAV'20
  \item \textbf{External Reviewer} -- ICALP'18, LICS'19, LICS'20, LICS'21, LICS'22, ESOP'20, ESOP'21, POPL'22, FoSSaCS'22
\end{itemize}

%\section{Technical and Personal Skills}
%
%\begin{itemize}
%
%\item \textbf{Programming Languages:} Proficient in: C, C++, Java, OCaml, Standard ML, Racket, Ruby, Rust; \\ Also familiar with: Coq, Scala, Python, JavaScript, Go, Objective-C.
%
%\vspace{4pt}
%
%\item \textbf{Softwares and Tools:} Rails, Soot, WALA, LLVM, Souffl{\'e}, Z3, Gurobi, PyTorch, Git, Vim, Emacs.
%
%\vspace{4pt}
%
%\item \textbf{Languages:} Native in Chinese; Fluent in English.
%
%\end{itemize}

\section{Scholarships and Awards}

\begin{itemize}
\item China National Scholarship \hfill 2014, 2016
\item Huawei Scholarship \hfill 2015

  \vspace{4pt}

  \item Silver Medal (5\textsuperscript{th} place) in the 39\textsuperscript{th} Annual ACM-ICPC World Finals \hfill 2015
  \item Gold Medal (1\textsuperscript{st} place) in the 39\textsuperscript{th} ACM-ICPC Asia Regionals Anshan site \hfill 2014
  \item Gold Medal (9\textsuperscript{th} place) in the 38\textsuperscript{th} ACM-ICPC Asia Regionals Changchun site \hfill 2013
\end{itemize}

\section{Talks}

\subsection{Conference Presentations}

\begin{itemize}
  \item {Sound Probabilistic Inference via Guide Types}, \emph{PLDI'21}. \hfill Jun 2021
  \item {Central Moment Analysis for Cost Accumulators in Probabilistic Programs}, \emph{PLDI'21}. \hfill Jun 2021
  \item {Raising Expectations: Automating Expected Cost Analysis with Types}, \emph{ICFP'20}. \hfill Aug 2020
  \item {Liquid Resource Types}, \emph{ICFP'20}. \hfill Aug 2020
  \item {A Denotational Semantics for Low-Level Probabilistic Programs with Nondeterminism}, \emph{MFPS'19}. \hfill Jun 2019
  \item {Type-Guided Worst-Case Input Generation}, \emph{POPL'19}. \hfill Jan 2019
  \item {PMAF: An Algebraic Framework for Static Analysis of Probabilistic Programs}, \emph{PLDI'18}. \hfill Jun 2018
\end{itemize}

\subsection{Seminar Presentations}

\begin{itemize}
  \item {Type-Based Resource-Guided Search}, \emph{Peking University}, Programming Language Seminar. \hfill Oct 2020
  \item {Taint Analysis for Blockchain Code}, \emph{Facebook}, Novice Seminar. \hfill Aug 2020
  \item {Automating Expected Cost Analysis with Types}, \emph{Facebook}, Novice Seminar. \hfill Jun 2020
\end{itemize}

\section{Projects}

%\begin{itemize}
  {\cventry{May 2020 -- Aug 2020}{Research Intern at Facebook}{Static Tag Analysis of Rust Code}{}{}{
      \begin{itemize}
        \item Studied the formal semantics of Rust and the static analysis tool MIRAI.
        \item Proposed and implemented a static tag analysis for Rust; the analysis keeps track of inter-procedural information flow, and allows user to customize tag propagation behavior of primitive operations.
        \item Applied the static tag analysis to analyze side-channel vulnerabilities of blockchain code.
      \end{itemize}
    }}

    \vspace{4pt}

  {\cventry{Feb 2018 -- May 2018}{Optimizing Compilers for Modern
        Architectures}{SIMD Vectorization in In-Memory DBMSs for OLAP
        Applications}{}{Carnegie Mellon University}{
        \begin{itemize}
        \item Proposed an optimization that uses vectorization in just-in-time
          query compilation.
        \item Implemented two approaches that use LLVM to emit SIMD instructions
          to vectorize predicate evaluation in Peloton, an in-memory DBMS
          developed by Carnegie Mellon Database Group.
        \item Achieved a significant speedup (avg. 1.5$\times$) on complex SQL queries.
        \end{itemize}
      }}

      \vspace{4pt}

  {\cventry{Apr 2018 -- May 2018}{Graduate Artificial Intelligence}{Predicting the Efficiency of Exact Inference Methods in Bayesian Network}{}{Carnegie Mellon University}{
        \begin{itemize}
        \item Reviewed exact inference methods for Bayesian networks from both the statistics and the programming languages community.
        \item Proposed and implemented a machine-learning-based algorithm that predicts which exact inference method would work best on a given Bayesian network.
        \item Achieved 72\% prediction accuracy on a synthetic test set.
        \end{itemize}
      }}

%      \vspace{4pt}

%    \item{\cventry{Feb 2017 -- Jun 2017}{Diploma Thesis}{ConDlog: Accelerating Program Analyses by Conditional Summarization with Datalog}{}{}{
%      \begin{itemize}
%        \item Researched on library summarization techniques for static analysis of large Java applications.
%        \item Proposed a Datalog-based algorithm that computes a complete analysis-specific conditional summary of JDK.
%        \item Achieved 3.9$\times$ and 1.3$\times$ speedup for Class Hierarchy Analysis and Rapid Type Analysis, respectively, on DaCapo benchmarks.
%      \end{itemize}
%    }}

%  \item{\cventry{Mar 2016 -- Jun 2016}{Design Principles of Programming Languages}{Melon: A Language
%        with Indexed Types}{}{}{
%        \begin{itemize}
%        \item Researched on type systems that can statically ensure dimension correctness
%          in matrix manipulation.
%        \item Implemented a language with a refinement type system that indexes matrices
%          by their dimensions and generates verification conditions for each
%          matrix operation.
%        \item Conducted a case study on linear regression and wrote a techical report that formally presents the language specification.
%        \end{itemize}
%      }}

%      \vspace{4pt}
%
%  \item{\cventry{Sep 2015 -- Jan 2016}{Software Analysis}{JStype: A Static
%        Analysis Tool for JavaScript}{}{}{
%        \begin{itemize}
%        \item Researched on static-analysis techniques for dynamic programming
%          languages, especially for JavaScript.
%        \item Implemented an abstract interpretater for a
%          CESK-style semantics of JavaScript with tunable analysis sensitivity.
%        \item Evaluated precision and efficiency of the analyzer on a broad suite of JavaScript programs.
%        \end{itemize}
%      }}

%      \vspace{4pt}

  % \item{\cventry{Nov 2015 -- Jan 2016}{Operating Systems (Honor
  %       Track)}{User-Level Thread Library}{}{}{
  %       \begin{itemize}
  %       \item Implemented a user-level thread library with a preemptive
  %         round-robin scheduler.
  %       \end{itemize}
  %     }}

%  {\cventry{Jul 2015 -- Aug 2015}{Querying Big Data: Theory and
%        Practice}{Parallel Scalability}{}{Peking University}{
%        \begin{itemize}
%        \item Proposed a new definition of linear/parallel scalability for the case where the
%          number of cores is far less than the scale of input, and conducted
%          theoretical analysis on several classical problems.
%        \item Proved graph reachability is not linear scalable, and not parallel scalable to some extent.
%        \item Wrote a technical paper that formally presents the results.
%        \end{itemize}
%      }}

  % \item{\cventry{Mar 2014 -- Jun 2014}{Linux Programming}{OItester: A Judge
  %       System for Programming Contests}{}{}{
  %       \begin{itemize}
  %       \item Designed and Implemented a web-based judge system with a separation
  %         between the front-end and judgers.
  %       \end{itemize}
  %     }}

% \section{References}

% \begin{center}
% \begin{tabular}{c @{\hspace{3em}} c @{\hspace{3em}} c}
% \large\textbf{Jan Hoffmann} & \large\textbf{Thomas Reps} & \large\textbf{Nadia Polikarpova} \\
% Carnegie Mellon University & University of Wisconsin--Madison & University of California, San Diego \\
% \emaillink{jhoffmann@cmu.edu} & \emaillink{reps@cs.wisc.edu} & \emaillink{npolikarpova@eng.ucsd.edu}
% \end{tabular}
% \end{center}


%-----       letter       ---------------------------------------------------------

\end{document}
