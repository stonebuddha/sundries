%% Copyright 2006-2013 Xavier Danaux (xdanaux@gmail.com).
%
% This work may be distributed and/or modified under the
% conditions of the LaTeX Project Public License version 1.3c,
% available at http://www.latex-project.org/lppl/.


\documentclass[11pt,letterpaper,sans]{moderncv}        % possible options include font size ('10pt', '11pt' and '12pt'), paper size ('a4paper', 'letterpaper', 'a5paper', 'legalpaper', 'executivepaper' and 'landscape') and font family ('sans' and 'roman')

% modern themes
\moderncvstyle{banking}                            % style options are 'casual' (default), 'classic', 'oldstyle' and 'banking'
\moderncvcolor{blue}                                % color options 'blue' (default), 'orange', 'green', 'red', 'purple', 'grey' and 'black'
%\renewcommand{\familydefault}{\sfdefault}         % to set the default font; use '\sfdefault' for the default sans serif font, '\rmdefault' for the default roman one, or any tex font name
%\nopagenumbers{}                                  % uncomment to suppress automatic page numbering for CVs longer than one page

% character encoding
\usepackage[utf8]{inputenc}                       % if you are not using xelatex ou lualatex, replace by the encoding you are using
%\usepackage{CJKutf8}                              % if you need to use CJK to typeset your resume in Chinese, Japanese or Korean
\usepackage[tt=false,type1=true]{libertine}
\usepackage[varqu]{zi4}
\usepackage[libertine]{newtxmath}

% adjust the page margins
\usepackage[scale=0.85]{geometry}
%\setlength{\hintscolumnwidth}{3cm}                % if you want to change the width of the column with the dates
%\setlength{\makecvtitlenamewidth}{10cm}           % for the 'classic' style, if you want to force the width allocated to your name and avoid line breaks. be careful though, the length is normally calculated to avoid any overlap with your personal info; use this at your own typographical risks...

\usepackage{import}

% personal data
\name{Di}{Wang}
\title{Curriculum Vitae}                               % optional, remove / comment the line if not wanted
%\address{my address, line 1, line 2, line 3, postcode}{}{}% optional, remove / comment the line if not wanted; the "postcode city" and and "country" arguments can be omitted or provided empty
\phone[mobile]{+1 646 371 6916}                   % optional, remove / comment the line if not wanted
%\phone[fixed]{01234 123456}                    % optional, remove / comment the line if not wanted
%\phone[fax]{+3~(456)~789~012}                      % optional, remove / comment the line if not wanted
\email{diw3@cs.cmu.edu}                               % optional, remove / comment the line if not wanted
\homepage{www.cs.cmu.edu/{\textasciitilde}diw3}                         % optional, remove / comment the line if not wanted
\extrainfo{Carnegie Mellon University -- Computer Science Department -- Doctoral Student}                 % optional, remove / comment the line if not wanted
%\photo[64pt][0.4pt]{picture}                       % optional, remove / comment the line if not wanted; '64pt' is the height the picture must be resized to, 0.4pt is the thickness of the frame around it (put it to 0pt for no frame) and 'picture' is the name of the picture file
%\quote{Some quote}                                 % optional, remove / comment the line if not wanted

% to show numerical labels in the bibliography (default is to show no labels); only useful if you make citations in your resume
\makeatletter
\renewcommand*{\bibliographyitemlabel}{\@biblabel{\arabic{enumiv}}}
\makeatother
%\renewcommand*{\bibliographyitemlabel}{[\arabic{enumiv}]}% CONSIDER REPLACING THE ABOVE BY THIS

% bibliography with mutiple entries
\usepackage{multibib}
\newcites{conf,thesis,draft}{{In Peer-Reviewed Conferences},{Theses},{Working Papers}}
%----------------------------------------------------------------------------------
%            content
%----------------------------------------------------------------------------------
\begin{document}
%\begin{CJK*}{UTF8}{gbsn}                          % to typeset your resume in Chinese using CJK
%-----       resume       ---------------------------------------------------------
\makecvtitle

%\small{Undergraduate electrical and electronic engineer completing the final year of a master's degree. Passionate about science, with strong technical, business, and interpersonal skills for working in a team and successfully completing a project.}

\vspace{-3em}

\section{Research Interests}

My research interests are in \emph{programming languages} and \emph{software engineering} with a focus on \emph{probabilistic programming}, \emph{quantitative program analysis}, \emph{higher type theory}, and \emph{program synthesis}.

\section{Education}

\begin{itemize}
	\item{\cventry{Aug 2017 -- present}{Doctoral Student}{Carnegie Mellon
        University}{Pittsburgh, PA}{}{Advisor: Prof. Jan Hoffmann}}
	
	\vspace{5pt}
	
	\item{\cventry{Sep 2013 -- Jun 2017}{B.S. (Hons) in Computer Science \&
        Technology}{Peking University}{Beijing, China}{}{Advisor: Prof. Yingfei
        Xiong}}
\end{itemize}

\section{Positions}

\begin{itemize}
	\item{\cventry{Aug 2017 -- present}{Research assistant in the group of Prof. Jan Hoffmann}{Carnegie Mellon University}{Pittsburgh, PA}{}{Topics: Static Analysis of Probabilistic Programs, Automatic Resource Bound Analysis}}
	
	\vspace{5pt}
	
	\item{\cventry{Sep 2016 -- Jan 2017}{Research intern in the group of Prof. Adam Chlipala}{Massachusetts Institute of Technology}{Boston, MA}{}{Topics: Type System for Complexity Analysis, Complexity Preserved Compiler}}
	
	\vspace{5pt}
	
	\item{\cventry{Jun 2016 -- Aug 2016}{Research intern in the group of Prof.
        Thomas Reps}{University of Wisconsin-Madison}{Madison, WI}{}{Topics:
        Probabilistic Reasoning about Side Channel Attacks, Expectation Invariant Analysis of Probabilistic Programs}}
	
	\vspace{5pt}
	
	\item{\cventry{Sep 2015 -- Jun 2017}{Research assistant in the group of Prof. Lu Zhang and Prof. Yingfei Xiong}{Peking University}{Beijing, China}{}{Topics: Complete Library Summarization for Program Analyses, Pointer Analysis for Java}}
\end{itemize}

\section{Technical and Personal Skills}

\begin{itemize}

\item \textbf{Programming Languages:} Proficient in: C, C++, Java, OCaml, Standard ML, Ruby; \\ Also familiar with: Rust, Coq, Agda, Racket, Python, JavaScript.

\vspace{5pt}

\item \textbf{Softwares and Tools:} Rails, Soot, WALA, Chord, LLVM, Souffl{\'e}, Git, Vim, Emacs.

\vspace{5pt}

\item \textbf{Languages:} Native in Chinese; Fluent in English.

\end{itemize}

\section{Scholarships and Awards}

\begin{itemize}
\item China National Scholarship \hfill 2014, 2016
\item Huawei Scholarship \hfill 2015
	\item Silver Medal (5\textsuperscript{th} place) in the 39\textsuperscript{th} Annual ACM-ICPC World Finals \hfill 2015
	\item Gold Medal (1\textsuperscript{st} place) in the 39\textsuperscript{th} ACM-ICPC Asia Regionals Anshan site \hfill 2014
	\item Gold Medal (9\textsuperscript{th} place) in the 38\textsuperscript{th} ACM-ICPC Asia Regionals Changchun site \hfill 2013
\end{itemize}

\section{Curriculum Projects}

\begin{itemize}
  \item{\cventry{Feb 2017 -- May 2017}{Optimizing Compilers for Modern
        Architectures}{SIMD Vectorization in In-Memory DBMSs for OLAP
        Applications}{}{}{
        \begin{itemize}
        \item Proposed an optimization that uses vectorization in just-in-time
          query compilation.
        \item Implemented two approaches that use LLVM to emit SIMD instructions
          to vectorize predicate evaluation in Peloton, an in-memory DBMS
          developed by Carnegie Mellon Database Group.
        \end{itemize}
      }}
  
  \item{\cventry{Mar 2016 -- Jun 2016}{Design Principles of Programming Languages}{Melon: A Language
        with Indexed Types}{}{}{
        \begin{itemize}
        \item Researched on type systems that can statically ensure dimension correctness
          in matrix manipulation.
        \item Implemented a language with a refinement type system that indexes matrices
          by their dimensions and generates verification conditions for each
          matrix operation.
        \end{itemize}
      }}

  \item{\cventry{Sep 2015 -- Jan 2016}{Software Analysis}{JStype: A Static
        Analysis Tool for JavaScript}{}{}{
        \begin{itemize}
        \item Researched on static analysis techniques for dynamic programming
          languages, especially for JavaScript.
        \item Implemented an abstract interpretation based static analyzer for a
          CESK-style semantics of JavaScript with tunable analysis sensitivity.
        \end{itemize}
      }}

  % \item{\cventry{Nov 2015 -- Jan 2016}{Operating Systems (Honor
  %       Track)}{User-Level Thread Library}{}{}{
  %       \begin{itemize}
  %       \item Implemented a user-level thread library with a preemptive
  %         round-robin scheduler.
  %       \end{itemize}
  %     }}

  \item{\cventry{Jul 2015 -- Aug 2015}{Querying Big Data: Thory and
        Practice}{Parallel Scalability}{}{}{
        \begin{itemize}
        \item Proposed a new definition of linear/parallel scalability when the
          number of cores is far less than the scale of input, and conducted
          theoretical analysis on several classical problems.
        \item Proved graph reachability is not linear scalable, and not parallel scalable to some extent.
        \end{itemize}
      }}

  % \item{\cventry{Mar 2014 -- Jun 2014}{Linux Programming}{OItester: A Judge
  %       System for Programming Contests}{}{}{
  %       \begin{itemize}
  %       \item Designed and Implemented a web-based judge system with a separation
  %         between the front-end and judgers.
  %       \end{itemize}
  %     }}
\end{itemize}



% Publications from a BibTeX file without multibib
%  for numerical labels: \renewcommand{\bibliographyitemlabel}{\@biblabel{\arabic{enumiv}}}% CONSIDER MERGING WITH PREAMBLE PART
%  to redefine the heading string ("Publications"): \renewcommand{\refname}{Articles}
%\nocite{*}
%\bibliographystyle{plain}
%\bibliography{publications}                        % 'publications' is the name of a BibTeX file

% Publications from a BibTeX file using the multibib package
\section{Publications}
\nociteconf{POPL:WH19,PLDI:WHR18,OOPSLA:WWC17,ESOP:TWX17}
\bibliographystyleconf{unsrt}
\vspace{5pt}
\bibliographyconf{publications}                   % 'publications' is the name of a BibTeX file
\nocitethesis{BS}
\bibliographystylethesis{unsrt}
\bibliographythesis{publications}
\vspace{5pt}
\nocitedraft{misc:KWP18,misc:WWC18,misc:WHR18}
\bibliographystyledraft{unsrt}
\bibliographydraft{publications}
%\nocitemisc{misc1,misc2,misc3}
%\bibliographystylemisc{plain}
%\bibliographymisc{publications}                   % 'publications' is the name of a BibTeX file

\section{Professional Activities}

\begin{itemize}
	\item Artifact Evaluation Committee -- Principles of Programming Languages (POPL'19) \hfill 2018
\end{itemize}


%-----       letter       ---------------------------------------------------------

\end{document}
